
% Default to the notebook output style

    


% Inherit from the specified cell style.




    
\documentclass[11pt]{article}

    
    
    \usepackage[T1]{fontenc}
    % Nicer default font (+ math font) than Computer Modern for most use cases
    \usepackage{mathpazo}

    % Basic figure setup, for now with no caption control since it's done
    % automatically by Pandoc (which extracts ![](path) syntax from Markdown).
    \usepackage{graphicx}
    % We will generate all images so they have a width \maxwidth. This means
    % that they will get their normal width if they fit onto the page, but
    % are scaled down if they would overflow the margins.
    \makeatletter
    \def\maxwidth{\ifdim\Gin@nat@width>\linewidth\linewidth
    \else\Gin@nat@width\fi}
    \makeatother
    \let\Oldincludegraphics\includegraphics
    % Set max figure width to be 80% of text width, for now hardcoded.
    \renewcommand{\includegraphics}[1]{\Oldincludegraphics[width=.8\maxwidth]{#1}}
    % Ensure that by default, figures have no caption (until we provide a
    % proper Figure object with a Caption API and a way to capture that
    % in the conversion process - todo).
    \usepackage{caption}
    \DeclareCaptionLabelFormat{nolabel}{}
    \captionsetup{labelformat=nolabel}

    \usepackage{adjustbox} % Used to constrain images to a maximum size 
    \usepackage{xcolor} % Allow colors to be defined
    \usepackage{enumerate} % Needed for markdown enumerations to work
    \usepackage{geometry} % Used to adjust the document margins
    \usepackage{amsmath} % Equations
    \usepackage{amssymb} % Equations
    \usepackage{textcomp} % defines textquotesingle
    % Hack from http://tex.stackexchange.com/a/47451/13684:
    \AtBeginDocument{%
        \def\PYZsq{\textquotesingle}% Upright quotes in Pygmentized code
    }
    \usepackage{upquote} % Upright quotes for verbatim code
    \usepackage{eurosym} % defines \euro
    \usepackage[mathletters]{ucs} % Extended unicode (utf-8) support
    \usepackage[utf8x]{inputenc} % Allow utf-8 characters in the tex document
    \usepackage{fancyvrb} % verbatim replacement that allows latex
    \usepackage{grffile} % extends the file name processing of package graphics 
                         % to support a larger range 
    % The hyperref package gives us a pdf with properly built
    % internal navigation ('pdf bookmarks' for the table of contents,
    % internal cross-reference links, web links for URLs, etc.)
    \usepackage{hyperref}
    \usepackage{longtable} % longtable support required by pandoc >1.10
    \usepackage{booktabs}  % table support for pandoc > 1.12.2
    \usepackage[inline]{enumitem} % IRkernel/repr support (it uses the enumerate* environment)
    \usepackage[normalem]{ulem} % ulem is needed to support strikethroughs (\sout)
                                % normalem makes italics be italics, not underlines
    

    
    
    % Colors for the hyperref package
    \definecolor{urlcolor}{rgb}{0,.145,.698}
    \definecolor{linkcolor}{rgb}{.71,0.21,0.01}
    \definecolor{citecolor}{rgb}{.12,.54,.11}

    % ANSI colors
    \definecolor{ansi-black}{HTML}{3E424D}
    \definecolor{ansi-black-intense}{HTML}{282C36}
    \definecolor{ansi-red}{HTML}{E75C58}
    \definecolor{ansi-red-intense}{HTML}{B22B31}
    \definecolor{ansi-green}{HTML}{00A250}
    \definecolor{ansi-green-intense}{HTML}{007427}
    \definecolor{ansi-yellow}{HTML}{DDB62B}
    \definecolor{ansi-yellow-intense}{HTML}{B27D12}
    \definecolor{ansi-blue}{HTML}{208FFB}
    \definecolor{ansi-blue-intense}{HTML}{0065CA}
    \definecolor{ansi-magenta}{HTML}{D160C4}
    \definecolor{ansi-magenta-intense}{HTML}{A03196}
    \definecolor{ansi-cyan}{HTML}{60C6C8}
    \definecolor{ansi-cyan-intense}{HTML}{258F8F}
    \definecolor{ansi-white}{HTML}{C5C1B4}
    \definecolor{ansi-white-intense}{HTML}{A1A6B2}

    % commands and environments needed by pandoc snippets
    % extracted from the output of `pandoc -s`
    \providecommand{\tightlist}{%
      \setlength{\itemsep}{0pt}\setlength{\parskip}{0pt}}
    \DefineVerbatimEnvironment{Highlighting}{Verbatim}{commandchars=\\\{\}}
    % Add ',fontsize=\small' for more characters per line
    \newenvironment{Shaded}{}{}
    \newcommand{\KeywordTok}[1]{\textcolor[rgb]{0.00,0.44,0.13}{\textbf{{#1}}}}
    \newcommand{\DataTypeTok}[1]{\textcolor[rgb]{0.56,0.13,0.00}{{#1}}}
    \newcommand{\DecValTok}[1]{\textcolor[rgb]{0.25,0.63,0.44}{{#1}}}
    \newcommand{\BaseNTok}[1]{\textcolor[rgb]{0.25,0.63,0.44}{{#1}}}
    \newcommand{\FloatTok}[1]{\textcolor[rgb]{0.25,0.63,0.44}{{#1}}}
    \newcommand{\CharTok}[1]{\textcolor[rgb]{0.25,0.44,0.63}{{#1}}}
    \newcommand{\StringTok}[1]{\textcolor[rgb]{0.25,0.44,0.63}{{#1}}}
    \newcommand{\CommentTok}[1]{\textcolor[rgb]{0.38,0.63,0.69}{\textit{{#1}}}}
    \newcommand{\OtherTok}[1]{\textcolor[rgb]{0.00,0.44,0.13}{{#1}}}
    \newcommand{\AlertTok}[1]{\textcolor[rgb]{1.00,0.00,0.00}{\textbf{{#1}}}}
    \newcommand{\FunctionTok}[1]{\textcolor[rgb]{0.02,0.16,0.49}{{#1}}}
    \newcommand{\RegionMarkerTok}[1]{{#1}}
    \newcommand{\ErrorTok}[1]{\textcolor[rgb]{1.00,0.00,0.00}{\textbf{{#1}}}}
    \newcommand{\NormalTok}[1]{{#1}}
    
    % Additional commands for more recent versions of Pandoc
    \newcommand{\ConstantTok}[1]{\textcolor[rgb]{0.53,0.00,0.00}{{#1}}}
    \newcommand{\SpecialCharTok}[1]{\textcolor[rgb]{0.25,0.44,0.63}{{#1}}}
    \newcommand{\VerbatimStringTok}[1]{\textcolor[rgb]{0.25,0.44,0.63}{{#1}}}
    \newcommand{\SpecialStringTok}[1]{\textcolor[rgb]{0.73,0.40,0.53}{{#1}}}
    \newcommand{\ImportTok}[1]{{#1}}
    \newcommand{\DocumentationTok}[1]{\textcolor[rgb]{0.73,0.13,0.13}{\textit{{#1}}}}
    \newcommand{\AnnotationTok}[1]{\textcolor[rgb]{0.38,0.63,0.69}{\textbf{\textit{{#1}}}}}
    \newcommand{\CommentVarTok}[1]{\textcolor[rgb]{0.38,0.63,0.69}{\textbf{\textit{{#1}}}}}
    \newcommand{\VariableTok}[1]{\textcolor[rgb]{0.10,0.09,0.49}{{#1}}}
    \newcommand{\ControlFlowTok}[1]{\textcolor[rgb]{0.00,0.44,0.13}{\textbf{{#1}}}}
    \newcommand{\OperatorTok}[1]{\textcolor[rgb]{0.40,0.40,0.40}{{#1}}}
    \newcommand{\BuiltInTok}[1]{{#1}}
    \newcommand{\ExtensionTok}[1]{{#1}}
    \newcommand{\PreprocessorTok}[1]{\textcolor[rgb]{0.74,0.48,0.00}{{#1}}}
    \newcommand{\AttributeTok}[1]{\textcolor[rgb]{0.49,0.56,0.16}{{#1}}}
    \newcommand{\InformationTok}[1]{\textcolor[rgb]{0.38,0.63,0.69}{\textbf{\textit{{#1}}}}}
    \newcommand{\WarningTok}[1]{\textcolor[rgb]{0.38,0.63,0.69}{\textbf{\textit{{#1}}}}}
    
    
    % Define a nice break command that doesn't care if a line doesn't already
    % exist.
    \def\br{\hspace*{\fill} \\* }
    % Math Jax compatability definitions
    \def\gt{>}
    \def\lt{<}
    % Document parameters
    \title{dream}
    
    
    

    % Pygments definitions
    
\makeatletter
\def\PY@reset{\let\PY@it=\relax \let\PY@bf=\relax%
    \let\PY@ul=\relax \let\PY@tc=\relax%
    \let\PY@bc=\relax \let\PY@ff=\relax}
\def\PY@tok#1{\csname PY@tok@#1\endcsname}
\def\PY@toks#1+{\ifx\relax#1\empty\else%
    \PY@tok{#1}\expandafter\PY@toks\fi}
\def\PY@do#1{\PY@bc{\PY@tc{\PY@ul{%
    \PY@it{\PY@bf{\PY@ff{#1}}}}}}}
\def\PY#1#2{\PY@reset\PY@toks#1+\relax+\PY@do{#2}}

\expandafter\def\csname PY@tok@nt\endcsname{\let\PY@bf=\textbf\def\PY@tc##1{\textcolor[rgb]{0.00,0.50,0.00}{##1}}}
\expandafter\def\csname PY@tok@vc\endcsname{\def\PY@tc##1{\textcolor[rgb]{0.10,0.09,0.49}{##1}}}
\expandafter\def\csname PY@tok@ge\endcsname{\let\PY@it=\textit}
\expandafter\def\csname PY@tok@nv\endcsname{\def\PY@tc##1{\textcolor[rgb]{0.10,0.09,0.49}{##1}}}
\expandafter\def\csname PY@tok@nb\endcsname{\def\PY@tc##1{\textcolor[rgb]{0.00,0.50,0.00}{##1}}}
\expandafter\def\csname PY@tok@c1\endcsname{\let\PY@it=\textit\def\PY@tc##1{\textcolor[rgb]{0.25,0.50,0.50}{##1}}}
\expandafter\def\csname PY@tok@c\endcsname{\let\PY@it=\textit\def\PY@tc##1{\textcolor[rgb]{0.25,0.50,0.50}{##1}}}
\expandafter\def\csname PY@tok@mo\endcsname{\def\PY@tc##1{\textcolor[rgb]{0.40,0.40,0.40}{##1}}}
\expandafter\def\csname PY@tok@fm\endcsname{\def\PY@tc##1{\textcolor[rgb]{0.00,0.00,1.00}{##1}}}
\expandafter\def\csname PY@tok@bp\endcsname{\def\PY@tc##1{\textcolor[rgb]{0.00,0.50,0.00}{##1}}}
\expandafter\def\csname PY@tok@vm\endcsname{\def\PY@tc##1{\textcolor[rgb]{0.10,0.09,0.49}{##1}}}
\expandafter\def\csname PY@tok@err\endcsname{\def\PY@bc##1{\setlength{\fboxsep}{0pt}\fcolorbox[rgb]{1.00,0.00,0.00}{1,1,1}{\strut ##1}}}
\expandafter\def\csname PY@tok@nd\endcsname{\def\PY@tc##1{\textcolor[rgb]{0.67,0.13,1.00}{##1}}}
\expandafter\def\csname PY@tok@vi\endcsname{\def\PY@tc##1{\textcolor[rgb]{0.10,0.09,0.49}{##1}}}
\expandafter\def\csname PY@tok@s2\endcsname{\def\PY@tc##1{\textcolor[rgb]{0.73,0.13,0.13}{##1}}}
\expandafter\def\csname PY@tok@vg\endcsname{\def\PY@tc##1{\textcolor[rgb]{0.10,0.09,0.49}{##1}}}
\expandafter\def\csname PY@tok@cpf\endcsname{\let\PY@it=\textit\def\PY@tc##1{\textcolor[rgb]{0.25,0.50,0.50}{##1}}}
\expandafter\def\csname PY@tok@sr\endcsname{\def\PY@tc##1{\textcolor[rgb]{0.73,0.40,0.53}{##1}}}
\expandafter\def\csname PY@tok@se\endcsname{\let\PY@bf=\textbf\def\PY@tc##1{\textcolor[rgb]{0.73,0.40,0.13}{##1}}}
\expandafter\def\csname PY@tok@ss\endcsname{\def\PY@tc##1{\textcolor[rgb]{0.10,0.09,0.49}{##1}}}
\expandafter\def\csname PY@tok@gi\endcsname{\def\PY@tc##1{\textcolor[rgb]{0.00,0.63,0.00}{##1}}}
\expandafter\def\csname PY@tok@gt\endcsname{\def\PY@tc##1{\textcolor[rgb]{0.00,0.27,0.87}{##1}}}
\expandafter\def\csname PY@tok@na\endcsname{\def\PY@tc##1{\textcolor[rgb]{0.49,0.56,0.16}{##1}}}
\expandafter\def\csname PY@tok@nc\endcsname{\let\PY@bf=\textbf\def\PY@tc##1{\textcolor[rgb]{0.00,0.00,1.00}{##1}}}
\expandafter\def\csname PY@tok@mh\endcsname{\def\PY@tc##1{\textcolor[rgb]{0.40,0.40,0.40}{##1}}}
\expandafter\def\csname PY@tok@sa\endcsname{\def\PY@tc##1{\textcolor[rgb]{0.73,0.13,0.13}{##1}}}
\expandafter\def\csname PY@tok@gu\endcsname{\let\PY@bf=\textbf\def\PY@tc##1{\textcolor[rgb]{0.50,0.00,0.50}{##1}}}
\expandafter\def\csname PY@tok@gh\endcsname{\let\PY@bf=\textbf\def\PY@tc##1{\textcolor[rgb]{0.00,0.00,0.50}{##1}}}
\expandafter\def\csname PY@tok@gs\endcsname{\let\PY@bf=\textbf}
\expandafter\def\csname PY@tok@cs\endcsname{\let\PY@it=\textit\def\PY@tc##1{\textcolor[rgb]{0.25,0.50,0.50}{##1}}}
\expandafter\def\csname PY@tok@mf\endcsname{\def\PY@tc##1{\textcolor[rgb]{0.40,0.40,0.40}{##1}}}
\expandafter\def\csname PY@tok@si\endcsname{\let\PY@bf=\textbf\def\PY@tc##1{\textcolor[rgb]{0.73,0.40,0.53}{##1}}}
\expandafter\def\csname PY@tok@sc\endcsname{\def\PY@tc##1{\textcolor[rgb]{0.73,0.13,0.13}{##1}}}
\expandafter\def\csname PY@tok@kp\endcsname{\def\PY@tc##1{\textcolor[rgb]{0.00,0.50,0.00}{##1}}}
\expandafter\def\csname PY@tok@s\endcsname{\def\PY@tc##1{\textcolor[rgb]{0.73,0.13,0.13}{##1}}}
\expandafter\def\csname PY@tok@ni\endcsname{\let\PY@bf=\textbf\def\PY@tc##1{\textcolor[rgb]{0.60,0.60,0.60}{##1}}}
\expandafter\def\csname PY@tok@sd\endcsname{\let\PY@it=\textit\def\PY@tc##1{\textcolor[rgb]{0.73,0.13,0.13}{##1}}}
\expandafter\def\csname PY@tok@mi\endcsname{\def\PY@tc##1{\textcolor[rgb]{0.40,0.40,0.40}{##1}}}
\expandafter\def\csname PY@tok@il\endcsname{\def\PY@tc##1{\textcolor[rgb]{0.40,0.40,0.40}{##1}}}
\expandafter\def\csname PY@tok@go\endcsname{\def\PY@tc##1{\textcolor[rgb]{0.53,0.53,0.53}{##1}}}
\expandafter\def\csname PY@tok@m\endcsname{\def\PY@tc##1{\textcolor[rgb]{0.40,0.40,0.40}{##1}}}
\expandafter\def\csname PY@tok@nn\endcsname{\let\PY@bf=\textbf\def\PY@tc##1{\textcolor[rgb]{0.00,0.00,1.00}{##1}}}
\expandafter\def\csname PY@tok@cp\endcsname{\def\PY@tc##1{\textcolor[rgb]{0.74,0.48,0.00}{##1}}}
\expandafter\def\csname PY@tok@sb\endcsname{\def\PY@tc##1{\textcolor[rgb]{0.73,0.13,0.13}{##1}}}
\expandafter\def\csname PY@tok@kt\endcsname{\def\PY@tc##1{\textcolor[rgb]{0.69,0.00,0.25}{##1}}}
\expandafter\def\csname PY@tok@ch\endcsname{\let\PY@it=\textit\def\PY@tc##1{\textcolor[rgb]{0.25,0.50,0.50}{##1}}}
\expandafter\def\csname PY@tok@sh\endcsname{\def\PY@tc##1{\textcolor[rgb]{0.73,0.13,0.13}{##1}}}
\expandafter\def\csname PY@tok@gr\endcsname{\def\PY@tc##1{\textcolor[rgb]{1.00,0.00,0.00}{##1}}}
\expandafter\def\csname PY@tok@kd\endcsname{\let\PY@bf=\textbf\def\PY@tc##1{\textcolor[rgb]{0.00,0.50,0.00}{##1}}}
\expandafter\def\csname PY@tok@cm\endcsname{\let\PY@it=\textit\def\PY@tc##1{\textcolor[rgb]{0.25,0.50,0.50}{##1}}}
\expandafter\def\csname PY@tok@no\endcsname{\def\PY@tc##1{\textcolor[rgb]{0.53,0.00,0.00}{##1}}}
\expandafter\def\csname PY@tok@gd\endcsname{\def\PY@tc##1{\textcolor[rgb]{0.63,0.00,0.00}{##1}}}
\expandafter\def\csname PY@tok@kr\endcsname{\let\PY@bf=\textbf\def\PY@tc##1{\textcolor[rgb]{0.00,0.50,0.00}{##1}}}
\expandafter\def\csname PY@tok@o\endcsname{\def\PY@tc##1{\textcolor[rgb]{0.40,0.40,0.40}{##1}}}
\expandafter\def\csname PY@tok@s1\endcsname{\def\PY@tc##1{\textcolor[rgb]{0.73,0.13,0.13}{##1}}}
\expandafter\def\csname PY@tok@kn\endcsname{\let\PY@bf=\textbf\def\PY@tc##1{\textcolor[rgb]{0.00,0.50,0.00}{##1}}}
\expandafter\def\csname PY@tok@nl\endcsname{\def\PY@tc##1{\textcolor[rgb]{0.63,0.63,0.00}{##1}}}
\expandafter\def\csname PY@tok@ow\endcsname{\let\PY@bf=\textbf\def\PY@tc##1{\textcolor[rgb]{0.67,0.13,1.00}{##1}}}
\expandafter\def\csname PY@tok@k\endcsname{\let\PY@bf=\textbf\def\PY@tc##1{\textcolor[rgb]{0.00,0.50,0.00}{##1}}}
\expandafter\def\csname PY@tok@mb\endcsname{\def\PY@tc##1{\textcolor[rgb]{0.40,0.40,0.40}{##1}}}
\expandafter\def\csname PY@tok@ne\endcsname{\let\PY@bf=\textbf\def\PY@tc##1{\textcolor[rgb]{0.82,0.25,0.23}{##1}}}
\expandafter\def\csname PY@tok@nf\endcsname{\def\PY@tc##1{\textcolor[rgb]{0.00,0.00,1.00}{##1}}}
\expandafter\def\csname PY@tok@kc\endcsname{\let\PY@bf=\textbf\def\PY@tc##1{\textcolor[rgb]{0.00,0.50,0.00}{##1}}}
\expandafter\def\csname PY@tok@gp\endcsname{\let\PY@bf=\textbf\def\PY@tc##1{\textcolor[rgb]{0.00,0.00,0.50}{##1}}}
\expandafter\def\csname PY@tok@w\endcsname{\def\PY@tc##1{\textcolor[rgb]{0.73,0.73,0.73}{##1}}}
\expandafter\def\csname PY@tok@dl\endcsname{\def\PY@tc##1{\textcolor[rgb]{0.73,0.13,0.13}{##1}}}
\expandafter\def\csname PY@tok@sx\endcsname{\def\PY@tc##1{\textcolor[rgb]{0.00,0.50,0.00}{##1}}}

\def\PYZbs{\char`\\}
\def\PYZus{\char`\_}
\def\PYZob{\char`\{}
\def\PYZcb{\char`\}}
\def\PYZca{\char`\^}
\def\PYZam{\char`\&}
\def\PYZlt{\char`\<}
\def\PYZgt{\char`\>}
\def\PYZsh{\char`\#}
\def\PYZpc{\char`\%}
\def\PYZdl{\char`\$}
\def\PYZhy{\char`\-}
\def\PYZsq{\char`\'}
\def\PYZdq{\char`\"}
\def\PYZti{\char`\~}
% for compatibility with earlier versions
\def\PYZat{@}
\def\PYZlb{[}
\def\PYZrb{]}
\makeatother


    % Exact colors from NB
    \definecolor{incolor}{rgb}{0.0, 0.0, 0.5}
    \definecolor{outcolor}{rgb}{0.545, 0.0, 0.0}



    
    % Prevent overflowing lines due to hard-to-break entities
    \sloppy 
    % Setup hyperref package
    \hypersetup{
      breaklinks=true,  % so long urls are correctly broken across lines
      colorlinks=true,
      urlcolor=urlcolor,
      linkcolor=linkcolor,
      citecolor=citecolor,
      }
    % Slightly bigger margins than the latex defaults
    
    \geometry{verbose,tmargin=1in,bmargin=1in,lmargin=1in,rmargin=1in}
    
    

    \begin{document}
    
    
    \maketitle
    
    

    
    \section{Deep Dreams (with Caffe)}\label{deep-dreams-with-caffe}

Credits: Forked from
\href{https://github.com/google/deepdream}{DeepDream} by Google

This notebook demonstrates how to use the
\href{http://caffe.berkeleyvision.org/}{Caffe} neural network framework
to produce "dream" visuals shown in the
\href{http://googleresearch.blogspot.ch/2015/06/inceptionism-going-deeper-into-neural.html}{Google
Research blog post}.

It'll be interesting to see what imagery people are able to generate
using the described technique. If you post images to Google+, Facebook,
or Twitter, be sure to tag them with \textbf{\#deepdream} so other
researchers can check them out too.

\subsection{Dependencies}\label{dependencies}

This notebook is designed to have as few dependencies as possible: *
Standard Python scientific stack: \href{http://www.numpy.org/}{NumPy},
\href{http://www.scipy.org/}{SciPy},
\href{http://www.pythonware.com/products/pil/}{PIL},
\href{http://ipython.org/}{IPython}. Those libraries can also be
installed as a part of one of the scientific packages for Python, such
as \href{http://continuum.io/downloads}{Anaconda} or
\href{https://store.enthought.com/}{Canopy}. *
\href{http://caffe.berkeleyvision.org/}{Caffe} deep learning framework
(\href{http://caffe.berkeleyvision.org/installation.html}{installation
instructions}). * Google
\href{https://developers.google.com/protocol-buffers/}{protobuf} library
that is used for Caffe model manipulation.

    \begin{Verbatim}[commandchars=\\\{\}]
{\color{incolor}In [{\color{incolor}1}]:} \PY{c+c1}{\PYZsh{} imports and basic notebook setup}
        \PY{k+kn}{from} \PY{n+nn}{io} \PY{k}{import} \PY{n}{StringIO}\PY{p}{,} \PY{n}{BytesIO}
        \PY{k+kn}{import} \PY{n+nn}{numpy} \PY{k}{as} \PY{n+nn}{np}
        \PY{k+kn}{import} \PY{n+nn}{scipy}\PY{n+nn}{.}\PY{n+nn}{ndimage} \PY{k}{as} \PY{n+nn}{nd}
        \PY{k+kn}{import} \PY{n+nn}{PIL}\PY{n+nn}{.}\PY{n+nn}{Image}
        \PY{k+kn}{from} \PY{n+nn}{IPython}\PY{n+nn}{.}\PY{n+nn}{display} \PY{k}{import} \PY{n}{clear\PYZus{}output}\PY{p}{,} \PY{n}{Image}\PY{p}{,} \PY{n}{display}
        \PY{k+kn}{from} \PY{n+nn}{google}\PY{n+nn}{.}\PY{n+nn}{protobuf} \PY{k}{import} \PY{n}{text\PYZus{}format}
        
        \PY{k+kn}{import} \PY{n+nn}{caffe}
        
        \PY{c+c1}{\PYZsh{} If your GPU supports CUDA and Caffe was built with CUDA support,}
        \PY{c+c1}{\PYZsh{} uncomment the following to run Caffe operations on the GPU.}
        \PY{c+c1}{\PYZsh{} caffe.set\PYZus{}mode\PYZus{}gpu()}
        \PY{c+c1}{\PYZsh{} caffe.set\PYZus{}device(0) \PYZsh{} select GPU device if multiple devices exist}
        
        
        \PY{k}{def} \PY{n+nf}{showarray}\PY{p}{(}\PY{n}{a}\PY{p}{,} \PY{n}{fmt}\PY{o}{=}\PY{l+s+s1}{\PYZsq{}}\PY{l+s+s1}{jpeg}\PY{l+s+s1}{\PYZsq{}}\PY{p}{)}\PY{p}{:}
            \PY{n}{a} \PY{o}{=} \PY{n}{np}\PY{o}{.}\PY{n}{uint8}\PY{p}{(}\PY{n}{np}\PY{o}{.}\PY{n}{clip}\PY{p}{(}\PY{n}{a}\PY{p}{,} \PY{l+m+mi}{0}\PY{p}{,} \PY{l+m+mi}{255}\PY{p}{)}\PY{p}{)}
        \PY{c+c1}{\PYZsh{}     f = StringIO()}
            \PY{n}{f} \PY{o}{=} \PY{n}{BytesIO}\PY{p}{(}\PY{p}{)}
            \PY{n}{PIL}\PY{o}{.}\PY{n}{Image}\PY{o}{.}\PY{n}{fromarray}\PY{p}{(}\PY{n}{a}\PY{p}{)}\PY{o}{.}\PY{n}{save}\PY{p}{(}\PY{n}{f}\PY{p}{,} \PY{n}{fmt}\PY{p}{)}
            \PY{n}{display}\PY{p}{(}\PY{n}{Image}\PY{p}{(}\PY{n}{data}\PY{o}{=}\PY{n}{f}\PY{o}{.}\PY{n}{getvalue}\PY{p}{(}\PY{p}{)}\PY{p}{)}\PY{p}{)}
\end{Verbatim}


    \subsection{Loading DNN model}\label{loading-dnn-model}

In this notebook we are going to use a
\href{https://github.com/BVLC/caffe/tree/master/models/bvlc_googlenet}{GoogLeNet}
model trained on \href{http://www.image-net.org/}{ImageNet} dataset.
Feel free to experiment with other models from Caffe
\href{https://github.com/BVLC/caffe/wiki/Model-Zoo}{Model Zoo}. One
particularly interesting
\href{http://places.csail.mit.edu/downloadCNN.html}{model} was trained
in \href{http://places.csail.mit.edu/}{MIT Places} dataset. It produced
many visuals from the
\href{http://googleresearch.blogspot.ch/2015/06/inceptionism-going-deeper-into-neural.html}{original
blog post}.

    \begin{Verbatim}[commandchars=\\\{\}]
{\color{incolor}In [{\color{incolor}2}]:} \PY{n}{model\PYZus{}path} \PY{o}{=} \PY{l+s+s1}{\PYZsq{}}\PY{l+s+s1}{./caffe/models/bvlc\PYZus{}googlenet/}\PY{l+s+s1}{\PYZsq{}} \PY{c+c1}{\PYZsh{} substitute your path here}
        \PY{n}{net\PYZus{}fn}   \PY{o}{=} \PY{n}{model\PYZus{}path} \PY{o}{+} \PY{l+s+s1}{\PYZsq{}}\PY{l+s+s1}{deploy.prototxt}\PY{l+s+s1}{\PYZsq{}}
        \PY{n}{param\PYZus{}fn} \PY{o}{=} \PY{n}{model\PYZus{}path} \PY{o}{+} \PY{l+s+s1}{\PYZsq{}}\PY{l+s+s1}{bvlc\PYZus{}googlenet.caffemodel}\PY{l+s+s1}{\PYZsq{}}
        
        \PY{c+c1}{\PYZsh{} Patching model to be able to compute gradients.}
        \PY{c+c1}{\PYZsh{} Note that you can also manually add \PYZdq{}force\PYZus{}backward: true\PYZdq{} line to \PYZdq{}deploy.prototxt\PYZdq{}.}
        \PY{n}{model} \PY{o}{=} \PY{n}{caffe}\PY{o}{.}\PY{n}{io}\PY{o}{.}\PY{n}{caffe\PYZus{}pb2}\PY{o}{.}\PY{n}{NetParameter}\PY{p}{(}\PY{p}{)}
        \PY{n}{text\PYZus{}format}\PY{o}{.}\PY{n}{Merge}\PY{p}{(}\PY{n+nb}{open}\PY{p}{(}\PY{n}{net\PYZus{}fn}\PY{p}{)}\PY{o}{.}\PY{n}{read}\PY{p}{(}\PY{p}{)}\PY{p}{,} \PY{n}{model}\PY{p}{)}
        \PY{n}{model}\PY{o}{.}\PY{n}{force\PYZus{}backward} \PY{o}{=} \PY{k+kc}{True}
        \PY{n+nb}{open}\PY{p}{(}\PY{l+s+s1}{\PYZsq{}}\PY{l+s+s1}{tmp.prototxt}\PY{l+s+s1}{\PYZsq{}}\PY{p}{,} \PY{l+s+s1}{\PYZsq{}}\PY{l+s+s1}{w}\PY{l+s+s1}{\PYZsq{}}\PY{p}{)}\PY{o}{.}\PY{n}{write}\PY{p}{(}\PY{n+nb}{str}\PY{p}{(}\PY{n}{model}\PY{p}{)}\PY{p}{)}
        
        \PY{n}{net} \PY{o}{=} \PY{n}{caffe}\PY{o}{.}\PY{n}{Classifier}\PY{p}{(}\PY{l+s+s1}{\PYZsq{}}\PY{l+s+s1}{tmp.prototxt}\PY{l+s+s1}{\PYZsq{}}\PY{p}{,} \PY{n}{param\PYZus{}fn}\PY{p}{,}
                               \PY{n}{mean} \PY{o}{=} \PY{n}{np}\PY{o}{.}\PY{n}{float32}\PY{p}{(}\PY{p}{[}\PY{l+m+mf}{104.0}\PY{p}{,} \PY{l+m+mf}{116.0}\PY{p}{,} \PY{l+m+mf}{122.0}\PY{p}{]}\PY{p}{)}\PY{p}{,} \PY{c+c1}{\PYZsh{} ImageNet mean, training set dependent}
                               \PY{n}{channel\PYZus{}swap} \PY{o}{=} \PY{p}{(}\PY{l+m+mi}{2}\PY{p}{,}\PY{l+m+mi}{1}\PY{p}{,}\PY{l+m+mi}{0}\PY{p}{)}\PY{p}{)} \PY{c+c1}{\PYZsh{} the reference model has channels in BGR order instead of RGB}
        
        \PY{c+c1}{\PYZsh{} a couple of utility functions for converting to and from Caffe\PYZsq{}s input image layout}
        \PY{k}{def} \PY{n+nf}{preprocess}\PY{p}{(}\PY{n}{net}\PY{p}{,} \PY{n}{img}\PY{p}{)}\PY{p}{:}
            \PY{k}{return} \PY{n}{np}\PY{o}{.}\PY{n}{float32}\PY{p}{(}\PY{n}{np}\PY{o}{.}\PY{n}{rollaxis}\PY{p}{(}\PY{n}{img}\PY{p}{,} \PY{l+m+mi}{2}\PY{p}{)}\PY{p}{[}\PY{p}{:}\PY{p}{:}\PY{o}{\PYZhy{}}\PY{l+m+mi}{1}\PY{p}{]}\PY{p}{)} \PY{o}{\PYZhy{}} \PY{n}{net}\PY{o}{.}\PY{n}{transformer}\PY{o}{.}\PY{n}{mean}\PY{p}{[}\PY{l+s+s1}{\PYZsq{}}\PY{l+s+s1}{data}\PY{l+s+s1}{\PYZsq{}}\PY{p}{]}
        \PY{k}{def} \PY{n+nf}{deprocess}\PY{p}{(}\PY{n}{net}\PY{p}{,} \PY{n}{img}\PY{p}{)}\PY{p}{:}
            \PY{k}{return} \PY{n}{np}\PY{o}{.}\PY{n}{dstack}\PY{p}{(}\PY{p}{(}\PY{n}{img} \PY{o}{+} \PY{n}{net}\PY{o}{.}\PY{n}{transformer}\PY{o}{.}\PY{n}{mean}\PY{p}{[}\PY{l+s+s1}{\PYZsq{}}\PY{l+s+s1}{data}\PY{l+s+s1}{\PYZsq{}}\PY{p}{]}\PY{p}{)}\PY{p}{[}\PY{p}{:}\PY{p}{:}\PY{o}{\PYZhy{}}\PY{l+m+mi}{1}\PY{p}{]}\PY{p}{)}
\end{Verbatim}


    \subsection{Producing dreams}\label{producing-dreams}

    Making the "dream" images is very simple. Essentially it is just a
gradient ascent process that tries to maximize the L2 norm of
activations of a particular DNN layer. Here are a few simple tricks that
we found useful for getting good images: * offset image by a random
jitter * normalize the magnitude of gradient ascent steps * apply ascent
across multiple scales (octaves)

First we implement a basic gradient ascent step function, applying the
first two tricks:

    \begin{Verbatim}[commandchars=\\\{\}]
{\color{incolor}In [{\color{incolor}3}]:} \PY{k}{def} \PY{n+nf}{objective\PYZus{}L2}\PY{p}{(}\PY{n}{dst}\PY{p}{)}\PY{p}{:}
            \PY{n}{dst}\PY{o}{.}\PY{n}{diff}\PY{p}{[}\PY{p}{:}\PY{p}{]} \PY{o}{=} \PY{n}{dst}\PY{o}{.}\PY{n}{data} 
        
        \PY{k}{def} \PY{n+nf}{make\PYZus{}step}\PY{p}{(}\PY{n}{net}\PY{p}{,} \PY{n}{step\PYZus{}size}\PY{o}{=}\PY{l+m+mf}{1.5}\PY{p}{,} \PY{n}{end}\PY{o}{=}\PY{l+s+s1}{\PYZsq{}}\PY{l+s+s1}{inception\PYZus{}4c/output}\PY{l+s+s1}{\PYZsq{}}\PY{p}{,} 
                      \PY{n}{jitter}\PY{o}{=}\PY{l+m+mi}{32}\PY{p}{,} \PY{n}{clip}\PY{o}{=}\PY{k+kc}{True}\PY{p}{,} \PY{n}{objective}\PY{o}{=}\PY{n}{objective\PYZus{}L2}\PY{p}{)}\PY{p}{:}
            \PY{l+s+sd}{\PYZsq{}\PYZsq{}\PYZsq{}Basic gradient ascent step.\PYZsq{}\PYZsq{}\PYZsq{}}
        
            \PY{n}{src} \PY{o}{=} \PY{n}{net}\PY{o}{.}\PY{n}{blobs}\PY{p}{[}\PY{l+s+s1}{\PYZsq{}}\PY{l+s+s1}{data}\PY{l+s+s1}{\PYZsq{}}\PY{p}{]} \PY{c+c1}{\PYZsh{} input image is stored in Net\PYZsq{}s \PYZsq{}data\PYZsq{} blob}
            \PY{n}{dst} \PY{o}{=} \PY{n}{net}\PY{o}{.}\PY{n}{blobs}\PY{p}{[}\PY{n}{end}\PY{p}{]}
        
            \PY{n}{ox}\PY{p}{,} \PY{n}{oy} \PY{o}{=} \PY{n}{np}\PY{o}{.}\PY{n}{random}\PY{o}{.}\PY{n}{randint}\PY{p}{(}\PY{o}{\PYZhy{}}\PY{n}{jitter}\PY{p}{,} \PY{n}{jitter}\PY{o}{+}\PY{l+m+mi}{1}\PY{p}{,} \PY{l+m+mi}{2}\PY{p}{)}
            \PY{n}{src}\PY{o}{.}\PY{n}{data}\PY{p}{[}\PY{l+m+mi}{0}\PY{p}{]} \PY{o}{=} \PY{n}{np}\PY{o}{.}\PY{n}{roll}\PY{p}{(}\PY{n}{np}\PY{o}{.}\PY{n}{roll}\PY{p}{(}\PY{n}{src}\PY{o}{.}\PY{n}{data}\PY{p}{[}\PY{l+m+mi}{0}\PY{p}{]}\PY{p}{,} \PY{n}{ox}\PY{p}{,} \PY{o}{\PYZhy{}}\PY{l+m+mi}{1}\PY{p}{)}\PY{p}{,} \PY{n}{oy}\PY{p}{,} \PY{o}{\PYZhy{}}\PY{l+m+mi}{2}\PY{p}{)} \PY{c+c1}{\PYZsh{} apply jitter shift}
                    
            \PY{n}{net}\PY{o}{.}\PY{n}{forward}\PY{p}{(}\PY{n}{end}\PY{o}{=}\PY{n}{end}\PY{p}{)}
            \PY{n}{objective}\PY{p}{(}\PY{n}{dst}\PY{p}{)}  \PY{c+c1}{\PYZsh{} specify the optimization objective}
            \PY{n}{net}\PY{o}{.}\PY{n}{backward}\PY{p}{(}\PY{n}{start}\PY{o}{=}\PY{n}{end}\PY{p}{)}
            \PY{n}{g} \PY{o}{=} \PY{n}{src}\PY{o}{.}\PY{n}{diff}\PY{p}{[}\PY{l+m+mi}{0}\PY{p}{]}
            \PY{c+c1}{\PYZsh{} apply normalized ascent step to the input image}
            \PY{n}{src}\PY{o}{.}\PY{n}{data}\PY{p}{[}\PY{p}{:}\PY{p}{]} \PY{o}{+}\PY{o}{=} \PY{n}{step\PYZus{}size}\PY{o}{/}\PY{n}{np}\PY{o}{.}\PY{n}{abs}\PY{p}{(}\PY{n}{g}\PY{p}{)}\PY{o}{.}\PY{n}{mean}\PY{p}{(}\PY{p}{)} \PY{o}{*} \PY{n}{g}
        
            \PY{n}{src}\PY{o}{.}\PY{n}{data}\PY{p}{[}\PY{l+m+mi}{0}\PY{p}{]} \PY{o}{=} \PY{n}{np}\PY{o}{.}\PY{n}{roll}\PY{p}{(}\PY{n}{np}\PY{o}{.}\PY{n}{roll}\PY{p}{(}\PY{n}{src}\PY{o}{.}\PY{n}{data}\PY{p}{[}\PY{l+m+mi}{0}\PY{p}{]}\PY{p}{,} \PY{o}{\PYZhy{}}\PY{n}{ox}\PY{p}{,} \PY{o}{\PYZhy{}}\PY{l+m+mi}{1}\PY{p}{)}\PY{p}{,} \PY{o}{\PYZhy{}}\PY{n}{oy}\PY{p}{,} \PY{o}{\PYZhy{}}\PY{l+m+mi}{2}\PY{p}{)} \PY{c+c1}{\PYZsh{} unshift image}
                    
            \PY{k}{if} \PY{n}{clip}\PY{p}{:}
                \PY{n}{bias} \PY{o}{=} \PY{n}{net}\PY{o}{.}\PY{n}{transformer}\PY{o}{.}\PY{n}{mean}\PY{p}{[}\PY{l+s+s1}{\PYZsq{}}\PY{l+s+s1}{data}\PY{l+s+s1}{\PYZsq{}}\PY{p}{]}
                \PY{n}{src}\PY{o}{.}\PY{n}{data}\PY{p}{[}\PY{p}{:}\PY{p}{]} \PY{o}{=} \PY{n}{np}\PY{o}{.}\PY{n}{clip}\PY{p}{(}\PY{n}{src}\PY{o}{.}\PY{n}{data}\PY{p}{,} \PY{o}{\PYZhy{}}\PY{n}{bias}\PY{p}{,} \PY{l+m+mi}{255}\PY{o}{\PYZhy{}}\PY{n}{bias}\PY{p}{)}    
\end{Verbatim}


    Next we implement an ascent through different scales. We call these
scales "octaves".

    \begin{Verbatim}[commandchars=\\\{\}]
{\color{incolor}In [{\color{incolor}5}]:} \PY{k}{def} \PY{n+nf}{deepdream}\PY{p}{(}\PY{n}{net}\PY{p}{,} \PY{n}{base\PYZus{}img}\PY{p}{,} \PY{n}{iter\PYZus{}n}\PY{o}{=}\PY{l+m+mi}{10}\PY{p}{,} \PY{n}{octave\PYZus{}n}\PY{o}{=}\PY{l+m+mi}{4}\PY{p}{,} \PY{n}{octave\PYZus{}scale}\PY{o}{=}\PY{l+m+mf}{1.4}\PY{p}{,} 
                      \PY{n}{end}\PY{o}{=}\PY{l+s+s1}{\PYZsq{}}\PY{l+s+s1}{inception\PYZus{}4c/output}\PY{l+s+s1}{\PYZsq{}}\PY{p}{,} \PY{n}{clip}\PY{o}{=}\PY{k+kc}{True}\PY{p}{,} \PY{o}{*}\PY{o}{*}\PY{n}{step\PYZus{}params}\PY{p}{)}\PY{p}{:}
            \PY{c+c1}{\PYZsh{} prepare base images for all octaves}
            \PY{n}{octaves} \PY{o}{=} \PY{p}{[}\PY{n}{preprocess}\PY{p}{(}\PY{n}{net}\PY{p}{,} \PY{n}{base\PYZus{}img}\PY{p}{)}\PY{p}{]}
            \PY{k}{for} \PY{n}{i} \PY{o+ow}{in} \PY{n+nb}{range}\PY{p}{(}\PY{n}{octave\PYZus{}n}\PY{o}{\PYZhy{}}\PY{l+m+mi}{1}\PY{p}{)}\PY{p}{:}
                \PY{n}{octaves}\PY{o}{.}\PY{n}{append}\PY{p}{(}\PY{n}{nd}\PY{o}{.}\PY{n}{zoom}\PY{p}{(}\PY{n}{octaves}\PY{p}{[}\PY{o}{\PYZhy{}}\PY{l+m+mi}{1}\PY{p}{]}\PY{p}{,} \PY{p}{(}\PY{l+m+mi}{1}\PY{p}{,} \PY{l+m+mf}{1.0}\PY{o}{/}\PY{n}{octave\PYZus{}scale}\PY{p}{,}\PY{l+m+mf}{1.0}\PY{o}{/}\PY{n}{octave\PYZus{}scale}\PY{p}{)}\PY{p}{,} \PY{n}{order}\PY{o}{=}\PY{l+m+mi}{1}\PY{p}{)}\PY{p}{)}
            
            \PY{n}{src} \PY{o}{=} \PY{n}{net}\PY{o}{.}\PY{n}{blobs}\PY{p}{[}\PY{l+s+s1}{\PYZsq{}}\PY{l+s+s1}{data}\PY{l+s+s1}{\PYZsq{}}\PY{p}{]}
            \PY{n}{detail} \PY{o}{=} \PY{n}{np}\PY{o}{.}\PY{n}{zeros\PYZus{}like}\PY{p}{(}\PY{n}{octaves}\PY{p}{[}\PY{o}{\PYZhy{}}\PY{l+m+mi}{1}\PY{p}{]}\PY{p}{)} \PY{c+c1}{\PYZsh{} allocate image for network\PYZhy{}produced details}
            \PY{k}{for} \PY{n}{octave}\PY{p}{,} \PY{n}{octave\PYZus{}base} \PY{o+ow}{in} \PY{n+nb}{enumerate}\PY{p}{(}\PY{n}{octaves}\PY{p}{[}\PY{p}{:}\PY{p}{:}\PY{o}{\PYZhy{}}\PY{l+m+mi}{1}\PY{p}{]}\PY{p}{)}\PY{p}{:}
                \PY{n}{h}\PY{p}{,} \PY{n}{w} \PY{o}{=} \PY{n}{octave\PYZus{}base}\PY{o}{.}\PY{n}{shape}\PY{p}{[}\PY{o}{\PYZhy{}}\PY{l+m+mi}{2}\PY{p}{:}\PY{p}{]}
                \PY{k}{if} \PY{n}{octave} \PY{o}{\PYZgt{}} \PY{l+m+mi}{0}\PY{p}{:}
                    \PY{c+c1}{\PYZsh{} upscale details from the previous octave}
                    \PY{n}{h1}\PY{p}{,} \PY{n}{w1} \PY{o}{=} \PY{n}{detail}\PY{o}{.}\PY{n}{shape}\PY{p}{[}\PY{o}{\PYZhy{}}\PY{l+m+mi}{2}\PY{p}{:}\PY{p}{]}
                    \PY{n}{detail} \PY{o}{=} \PY{n}{nd}\PY{o}{.}\PY{n}{zoom}\PY{p}{(}\PY{n}{detail}\PY{p}{,} \PY{p}{(}\PY{l+m+mi}{1}\PY{p}{,} \PY{l+m+mf}{1.0}\PY{o}{*}\PY{n}{h}\PY{o}{/}\PY{n}{h1}\PY{p}{,}\PY{l+m+mf}{1.0}\PY{o}{*}\PY{n}{w}\PY{o}{/}\PY{n}{w1}\PY{p}{)}\PY{p}{,} \PY{n}{order}\PY{o}{=}\PY{l+m+mi}{1}\PY{p}{)}
        
                \PY{n}{src}\PY{o}{.}\PY{n}{reshape}\PY{p}{(}\PY{l+m+mi}{1}\PY{p}{,}\PY{l+m+mi}{3}\PY{p}{,}\PY{n}{h}\PY{p}{,}\PY{n}{w}\PY{p}{)} \PY{c+c1}{\PYZsh{} resize the network\PYZsq{}s input image size}
                \PY{n}{src}\PY{o}{.}\PY{n}{data}\PY{p}{[}\PY{l+m+mi}{0}\PY{p}{]} \PY{o}{=} \PY{n}{octave\PYZus{}base}\PY{o}{+}\PY{n}{detail}
                \PY{k}{for} \PY{n}{i} \PY{o+ow}{in} \PY{n+nb}{range}\PY{p}{(}\PY{n}{iter\PYZus{}n}\PY{p}{)}\PY{p}{:}
                    \PY{n}{make\PYZus{}step}\PY{p}{(}\PY{n}{net}\PY{p}{,} \PY{n}{end}\PY{o}{=}\PY{n}{end}\PY{p}{,} \PY{n}{clip}\PY{o}{=}\PY{n}{clip}\PY{p}{,} \PY{o}{*}\PY{o}{*}\PY{n}{step\PYZus{}params}\PY{p}{)}
                    
                    \PY{c+c1}{\PYZsh{} visualization}
                    \PY{n}{vis} \PY{o}{=} \PY{n}{deprocess}\PY{p}{(}\PY{n}{net}\PY{p}{,} \PY{n}{src}\PY{o}{.}\PY{n}{data}\PY{p}{[}\PY{l+m+mi}{0}\PY{p}{]}\PY{p}{)}
                    \PY{k}{if} \PY{o+ow}{not} \PY{n}{clip}\PY{p}{:} \PY{c+c1}{\PYZsh{} adjust image contrast if clipping is disabled}
                        \PY{n}{vis} \PY{o}{=} \PY{n}{vis}\PY{o}{*}\PY{p}{(}\PY{l+m+mf}{255.0}\PY{o}{/}\PY{n}{np}\PY{o}{.}\PY{n}{percentile}\PY{p}{(}\PY{n}{vis}\PY{p}{,} \PY{l+m+mf}{99.98}\PY{p}{)}\PY{p}{)}
                    \PY{n}{showarray}\PY{p}{(}\PY{n}{vis}\PY{p}{)}
                    \PY{n+nb}{print}\PY{p}{(}\PY{n}{octave}\PY{p}{,} \PY{n}{i}\PY{p}{,} \PY{n}{end}\PY{p}{,} \PY{n}{vis}\PY{o}{.}\PY{n}{shape}\PY{p}{)}
                    \PY{n}{clear\PYZus{}output}\PY{p}{(}\PY{n}{wait}\PY{o}{=}\PY{k+kc}{True}\PY{p}{)}
                    
                \PY{c+c1}{\PYZsh{} extract details produced on the current octave}
                \PY{n}{detail} \PY{o}{=} \PY{n}{src}\PY{o}{.}\PY{n}{data}\PY{p}{[}\PY{l+m+mi}{0}\PY{p}{]}\PY{o}{\PYZhy{}}\PY{n}{octave\PYZus{}base}
            \PY{c+c1}{\PYZsh{} returning the resulting image}
            \PY{k}{return} \PY{n}{deprocess}\PY{p}{(}\PY{n}{net}\PY{p}{,} \PY{n}{src}\PY{o}{.}\PY{n}{data}\PY{p}{[}\PY{l+m+mi}{0}\PY{p}{]}\PY{p}{)}
\end{Verbatim}


    Now we are ready to let the neural network reveal its dreams! Let's take
a
\href{https://commons.wikimedia.org/wiki/File:Appearance_of_sky_for_weather_forecast,_Dhaka,_Bangladesh.JPG}{cloud
image} as a starting point:

    \begin{Verbatim}[commandchars=\\\{\}]
{\color{incolor}In [{\color{incolor}10}]:} \PY{n}{img} \PY{o}{=} \PY{n}{np}\PY{o}{.}\PY{n}{float32}\PY{p}{(}\PY{n}{PIL}\PY{o}{.}\PY{n}{Image}\PY{o}{.}\PY{n}{open}\PY{p}{(}\PY{l+s+s1}{\PYZsq{}}\PY{l+s+s1}{sky1024px.jpg}\PY{l+s+s1}{\PYZsq{}}\PY{p}{)}\PY{p}{)}
         \PY{n}{showarray}\PY{p}{(}\PY{n}{img}\PY{p}{)}
\end{Verbatim}


    \begin{center}
    \adjustimage{max size={0.9\linewidth}{0.9\paperheight}}{output_10_0.jpg}
    \end{center}
    { \hspace*{\fill} \\}
    
    Running the next code cell starts the detail generation process. You may
see how new patterns start to form, iteration by iteration, octave by
octave.

    \begin{Verbatim}[commandchars=\\\{\}]
{\color{incolor}In [{\color{incolor}18}]:} \PY{n}{\PYZus{}}\PY{o}{=}\PY{n}{deepdream}\PY{p}{(}\PY{n}{net}\PY{p}{,} \PY{n}{img}\PY{p}{)}
\end{Verbatim}


    \begin{center}
    \adjustimage{max size={0.9\linewidth}{0.9\paperheight}}{output_12_0.jpg}
    \end{center}
    { \hspace*{\fill} \\}
    
    \begin{Verbatim}[commandchars=\\\{\}]
3 9 inception\_4c/output (575, 1024, 3)

    \end{Verbatim}

    The complexity of the details generated depends on which layer's
activations we try to maximize. Higher layers produce complex features,
while lower ones enhance edges and textures, giving the image an
impressionist feeling:

    \begin{Verbatim}[commandchars=\\\{\}]
{\color{incolor}In [{\color{incolor}19}]:} \PY{n}{\PYZus{}}\PY{o}{=}\PY{n}{deepdream}\PY{p}{(}\PY{n}{net}\PY{p}{,} \PY{n}{img}\PY{p}{,} \PY{n}{end}\PY{o}{=}\PY{l+s+s1}{\PYZsq{}}\PY{l+s+s1}{inception\PYZus{}3b/5x5\PYZus{}reduce}\PY{l+s+s1}{\PYZsq{}}\PY{p}{)}
\end{Verbatim}


    \begin{center}
    \adjustimage{max size={0.9\linewidth}{0.9\paperheight}}{output_14_0.jpg}
    \end{center}
    { \hspace*{\fill} \\}
    
    \begin{Verbatim}[commandchars=\\\{\}]
3 9 inception\_3b/5x5\_reduce (575, 1024, 3)

    \end{Verbatim}

    We encourage readers to experiment with layer selection to see how it
affects the results. Execute the next code cell to see the list of
different layers. You can modify the \texttt{make\_step} function to
make it follow some different objective, say to select a subset of
activations to maximize, or to maximize multiple layers at once. There
is a huge design space to explore!

    \begin{Verbatim}[commandchars=\\\{\}]
{\color{incolor}In [{\color{incolor}20}]:} \PY{n}{net}\PY{o}{.}\PY{n}{blobs}\PY{o}{.}\PY{n}{keys}\PY{p}{(}\PY{p}{)}
\end{Verbatim}


\begin{Verbatim}[commandchars=\\\{\}]
{\color{outcolor}Out[{\color{outcolor}20}]:} odict\_keys(['data', 'conv1/7x7\_s2', 'pool1/3x3\_s2', 'pool1/norm1', 'conv2/3x3\_reduce', 'conv2/3x3', 'conv2/norm2', 'pool2/3x3\_s2', 'pool2/3x3\_s2\_pool2/3x3\_s2\_0\_split\_0', 'pool2/3x3\_s2\_pool2/3x3\_s2\_0\_split\_1', 'pool2/3x3\_s2\_pool2/3x3\_s2\_0\_split\_2', 'pool2/3x3\_s2\_pool2/3x3\_s2\_0\_split\_3', 'inception\_3a/1x1', 'inception\_3a/3x3\_reduce', 'inception\_3a/3x3', 'inception\_3a/5x5\_reduce', 'inception\_3a/5x5', 'inception\_3a/pool', 'inception\_3a/pool\_proj', 'inception\_3a/output', 'inception\_3a/output\_inception\_3a/output\_0\_split\_0', 'inception\_3a/output\_inception\_3a/output\_0\_split\_1', 'inception\_3a/output\_inception\_3a/output\_0\_split\_2', 'inception\_3a/output\_inception\_3a/output\_0\_split\_3', 'inception\_3b/1x1', 'inception\_3b/3x3\_reduce', 'inception\_3b/3x3', 'inception\_3b/5x5\_reduce', 'inception\_3b/5x5', 'inception\_3b/pool', 'inception\_3b/pool\_proj', 'inception\_3b/output', 'pool3/3x3\_s2', 'pool3/3x3\_s2\_pool3/3x3\_s2\_0\_split\_0', 'pool3/3x3\_s2\_pool3/3x3\_s2\_0\_split\_1', 'pool3/3x3\_s2\_pool3/3x3\_s2\_0\_split\_2', 'pool3/3x3\_s2\_pool3/3x3\_s2\_0\_split\_3', 'inception\_4a/1x1', 'inception\_4a/3x3\_reduce', 'inception\_4a/3x3', 'inception\_4a/5x5\_reduce', 'inception\_4a/5x5', 'inception\_4a/pool', 'inception\_4a/pool\_proj', 'inception\_4a/output', 'inception\_4a/output\_inception\_4a/output\_0\_split\_0', 'inception\_4a/output\_inception\_4a/output\_0\_split\_1', 'inception\_4a/output\_inception\_4a/output\_0\_split\_2', 'inception\_4a/output\_inception\_4a/output\_0\_split\_3', 'inception\_4b/1x1', 'inception\_4b/3x3\_reduce', 'inception\_4b/3x3', 'inception\_4b/5x5\_reduce', 'inception\_4b/5x5', 'inception\_4b/pool', 'inception\_4b/pool\_proj', 'inception\_4b/output', 'inception\_4b/output\_inception\_4b/output\_0\_split\_0', 'inception\_4b/output\_inception\_4b/output\_0\_split\_1', 'inception\_4b/output\_inception\_4b/output\_0\_split\_2', 'inception\_4b/output\_inception\_4b/output\_0\_split\_3', 'inception\_4c/1x1', 'inception\_4c/3x3\_reduce', 'inception\_4c/3x3', 'inception\_4c/5x5\_reduce', 'inception\_4c/5x5', 'inception\_4c/pool', 'inception\_4c/pool\_proj', 'inception\_4c/output', 'inception\_4c/output\_inception\_4c/output\_0\_split\_0', 'inception\_4c/output\_inception\_4c/output\_0\_split\_1', 'inception\_4c/output\_inception\_4c/output\_0\_split\_2', 'inception\_4c/output\_inception\_4c/output\_0\_split\_3', 'inception\_4d/1x1', 'inception\_4d/3x3\_reduce', 'inception\_4d/3x3', 'inception\_4d/5x5\_reduce', 'inception\_4d/5x5', 'inception\_4d/pool', 'inception\_4d/pool\_proj', 'inception\_4d/output', 'inception\_4d/output\_inception\_4d/output\_0\_split\_0', 'inception\_4d/output\_inception\_4d/output\_0\_split\_1', 'inception\_4d/output\_inception\_4d/output\_0\_split\_2', 'inception\_4d/output\_inception\_4d/output\_0\_split\_3', 'inception\_4e/1x1', 'inception\_4e/3x3\_reduce', 'inception\_4e/3x3', 'inception\_4e/5x5\_reduce', 'inception\_4e/5x5', 'inception\_4e/pool', 'inception\_4e/pool\_proj', 'inception\_4e/output', 'pool4/3x3\_s2', 'pool4/3x3\_s2\_pool4/3x3\_s2\_0\_split\_0', 'pool4/3x3\_s2\_pool4/3x3\_s2\_0\_split\_1', 'pool4/3x3\_s2\_pool4/3x3\_s2\_0\_split\_2', 'pool4/3x3\_s2\_pool4/3x3\_s2\_0\_split\_3', 'inception\_5a/1x1', 'inception\_5a/3x3\_reduce', 'inception\_5a/3x3', 'inception\_5a/5x5\_reduce', 'inception\_5a/5x5', 'inception\_5a/pool', 'inception\_5a/pool\_proj', 'inception\_5a/output', 'inception\_5a/output\_inception\_5a/output\_0\_split\_0', 'inception\_5a/output\_inception\_5a/output\_0\_split\_1', 'inception\_5a/output\_inception\_5a/output\_0\_split\_2', 'inception\_5a/output\_inception\_5a/output\_0\_split\_3', 'inception\_5b/1x1', 'inception\_5b/3x3\_reduce', 'inception\_5b/3x3', 'inception\_5b/5x5\_reduce', 'inception\_5b/5x5', 'inception\_5b/pool', 'inception\_5b/pool\_proj', 'inception\_5b/output', 'pool5/7x7\_s1', 'loss3/classifier', 'prob'])
\end{Verbatim}
            
    What if we feed the \texttt{deepdream} function its own output, after
applying a little zoom to it? It turns out that this leads to an endless
stream of impressions of the things that the network saw during
training. Some patterns fire more often than others, suggestive of
basins of attraction.

We will start the process from the same sky image as above, but after
some iteration the original image becomes irrelevant; even random noise
can be used as the starting point.

    \begin{Verbatim}[commandchars=\\\{\}]
{\color{incolor}In [{\color{incolor}21}]:} \PY{o}{!}mkdir frames
         \PY{n}{frame} \PY{o}{=} \PY{n}{img}
         \PY{n}{frame\PYZus{}i} \PY{o}{=} \PY{l+m+mi}{0}
\end{Verbatim}


    \begin{Verbatim}[commandchars=\\\{\}]
{\color{incolor}In [{\color{incolor} }]:} \PY{n}{h}\PY{p}{,} \PY{n}{w} \PY{o}{=} \PY{n}{frame}\PY{o}{.}\PY{n}{shape}\PY{p}{[}\PY{p}{:}\PY{l+m+mi}{2}\PY{p}{]}
        \PY{n}{s} \PY{o}{=} \PY{l+m+mf}{0.05} \PY{c+c1}{\PYZsh{} scale coefficient}
        \PY{k}{for} \PY{n}{i} \PY{o+ow}{in} \PY{n+nb}{range}\PY{p}{(}\PY{l+m+mi}{100}\PY{p}{)}\PY{p}{:}
            \PY{n}{frame} \PY{o}{=} \PY{n}{deepdream}\PY{p}{(}\PY{n}{net}\PY{p}{,} \PY{n}{frame}\PY{p}{)}
            \PY{n}{PIL}\PY{o}{.}\PY{n}{Image}\PY{o}{.}\PY{n}{fromarray}\PY{p}{(}\PY{n}{np}\PY{o}{.}\PY{n}{uint8}\PY{p}{(}\PY{n}{frame}\PY{p}{)}\PY{p}{)}\PY{o}{.}\PY{n}{save}\PY{p}{(}\PY{l+s+s2}{\PYZdq{}}\PY{l+s+s2}{frames/}\PY{l+s+si}{\PYZpc{}04d}\PY{l+s+s2}{.jpg}\PY{l+s+s2}{\PYZdq{}}\PY{o}{\PYZpc{}}\PY{k}{frame\PYZus{}i})
            \PY{n}{frame} \PY{o}{=} \PY{n}{nd}\PY{o}{.}\PY{n}{affine\PYZus{}transform}\PY{p}{(}\PY{n}{frame}\PY{p}{,} \PY{p}{[}\PY{l+m+mi}{1}\PY{o}{\PYZhy{}}\PY{n}{s}\PY{p}{,}\PY{l+m+mi}{1}\PY{o}{\PYZhy{}}\PY{n}{s}\PY{p}{,}\PY{l+m+mi}{1}\PY{p}{]}\PY{p}{,} \PY{p}{[}\PY{n}{h}\PY{o}{*}\PY{n}{s}\PY{o}{/}\PY{l+m+mi}{2}\PY{p}{,}\PY{n}{w}\PY{o}{*}\PY{n}{s}\PY{o}{/}\PY{l+m+mi}{2}\PY{p}{,}\PY{l+m+mi}{0}\PY{p}{]}\PY{p}{,} \PY{n}{order}\PY{o}{=}\PY{l+m+mi}{1}\PY{p}{)}
            \PY{n}{frame\PYZus{}i} \PY{o}{+}\PY{o}{=} \PY{l+m+mi}{1}
\end{Verbatim}


    \begin{center}
    \adjustimage{max size={0.9\linewidth}{0.9\paperheight}}{output_19_0.jpg}
    \end{center}
    { \hspace*{\fill} \\}
    
    \begin{Verbatim}[commandchars=\\\{\}]
3 4 inception\_4c/output (575, 1024, 3)

    \end{Verbatim}

    Be careful running the code above, it can bring you into very strange
realms!

    \begin{Verbatim}[commandchars=\\\{\}]
{\color{incolor}In [{\color{incolor}6}]:} \PY{n}{Image}\PY{p}{(}\PY{n}{filename}\PY{o}{=}\PY{l+s+s1}{\PYZsq{}}\PY{l+s+s1}{frames/0029.jpg}\PY{l+s+s1}{\PYZsq{}}\PY{p}{)}
\end{Verbatim}

\texttt{\color{outcolor}Out[{\color{outcolor}6}]:}
    
    \begin{center}
    \adjustimage{max size={0.9\linewidth}{0.9\paperheight}}{output_21_0.jpg}
    \end{center}
    { \hspace*{\fill} \\}
    

    \subsection{Controlling dreams}\label{controlling-dreams}

The image detail generation method described above tends to produce some
patterns more often the others. One easy way to improve the generated
image diversity is to tweak the optimization objective. Here we show
just one of many ways to do that. Let's use one more input image. We'd
call it a "\emph{guide}".

    \begin{Verbatim}[commandchars=\\\{\}]
{\color{incolor}In [{\color{incolor}7}]:} \PY{n}{guide} \PY{o}{=} \PY{n}{np}\PY{o}{.}\PY{n}{float32}\PY{p}{(}\PY{n}{PIL}\PY{o}{.}\PY{n}{Image}\PY{o}{.}\PY{n}{open}\PY{p}{(}\PY{l+s+s1}{\PYZsq{}}\PY{l+s+s1}{flowers.jpg}\PY{l+s+s1}{\PYZsq{}}\PY{p}{)}\PY{p}{)}
        \PY{n}{showarray}\PY{p}{(}\PY{n}{guide}\PY{p}{)}
\end{Verbatim}


    \begin{center}
    \adjustimage{max size={0.9\linewidth}{0.9\paperheight}}{output_23_0.jpg}
    \end{center}
    { \hspace*{\fill} \\}
    
    Note that the neural network we use was trained on images downscaled to
224x224 size. So high resolution images might have to be downscaled, so
that the network could pick up their features. The image we use here is
already small enough.

Now we pick some target layer and extract guide image features.

    \begin{Verbatim}[commandchars=\\\{\}]
{\color{incolor}In [{\color{incolor}8}]:} \PY{n}{end} \PY{o}{=} \PY{l+s+s1}{\PYZsq{}}\PY{l+s+s1}{inception\PYZus{}3b/output}\PY{l+s+s1}{\PYZsq{}}
        \PY{n}{h}\PY{p}{,} \PY{n}{w} \PY{o}{=} \PY{n}{guide}\PY{o}{.}\PY{n}{shape}\PY{p}{[}\PY{p}{:}\PY{l+m+mi}{2}\PY{p}{]}
        \PY{n}{src}\PY{p}{,} \PY{n}{dst} \PY{o}{=} \PY{n}{net}\PY{o}{.}\PY{n}{blobs}\PY{p}{[}\PY{l+s+s1}{\PYZsq{}}\PY{l+s+s1}{data}\PY{l+s+s1}{\PYZsq{}}\PY{p}{]}\PY{p}{,} \PY{n}{net}\PY{o}{.}\PY{n}{blobs}\PY{p}{[}\PY{n}{end}\PY{p}{]}
        \PY{n}{src}\PY{o}{.}\PY{n}{reshape}\PY{p}{(}\PY{l+m+mi}{1}\PY{p}{,}\PY{l+m+mi}{3}\PY{p}{,}\PY{n}{h}\PY{p}{,}\PY{n}{w}\PY{p}{)}
        \PY{n}{src}\PY{o}{.}\PY{n}{data}\PY{p}{[}\PY{l+m+mi}{0}\PY{p}{]} \PY{o}{=} \PY{n}{preprocess}\PY{p}{(}\PY{n}{net}\PY{p}{,} \PY{n}{guide}\PY{p}{)}
        \PY{n}{net}\PY{o}{.}\PY{n}{forward}\PY{p}{(}\PY{n}{end}\PY{o}{=}\PY{n}{end}\PY{p}{)}
        \PY{n}{guide\PYZus{}features} \PY{o}{=} \PY{n}{dst}\PY{o}{.}\PY{n}{data}\PY{p}{[}\PY{l+m+mi}{0}\PY{p}{]}\PY{o}{.}\PY{n}{copy}\PY{p}{(}\PY{p}{)}
\end{Verbatim}


    Instead of maximizing the L2-norm of current image activations, we try
to maximize the dot-products between activations of current image, and
their best matching correspondences from the guide image.

    \begin{Verbatim}[commandchars=\\\{\}]
{\color{incolor}In [{\color{incolor}11}]:} \PY{k}{def} \PY{n+nf}{objective\PYZus{}guide}\PY{p}{(}\PY{n}{dst}\PY{p}{)}\PY{p}{:}
             \PY{n}{x} \PY{o}{=} \PY{n}{dst}\PY{o}{.}\PY{n}{data}\PY{p}{[}\PY{l+m+mi}{0}\PY{p}{]}\PY{o}{.}\PY{n}{copy}\PY{p}{(}\PY{p}{)}
             \PY{n}{y} \PY{o}{=} \PY{n}{guide\PYZus{}features}
             \PY{n}{ch} \PY{o}{=} \PY{n}{x}\PY{o}{.}\PY{n}{shape}\PY{p}{[}\PY{l+m+mi}{0}\PY{p}{]}
             \PY{n}{x} \PY{o}{=} \PY{n}{x}\PY{o}{.}\PY{n}{reshape}\PY{p}{(}\PY{n}{ch}\PY{p}{,}\PY{o}{\PYZhy{}}\PY{l+m+mi}{1}\PY{p}{)}
             \PY{n}{y} \PY{o}{=} \PY{n}{y}\PY{o}{.}\PY{n}{reshape}\PY{p}{(}\PY{n}{ch}\PY{p}{,}\PY{o}{\PYZhy{}}\PY{l+m+mi}{1}\PY{p}{)}
             \PY{n}{A} \PY{o}{=} \PY{n}{x}\PY{o}{.}\PY{n}{T}\PY{o}{.}\PY{n}{dot}\PY{p}{(}\PY{n}{y}\PY{p}{)} \PY{c+c1}{\PYZsh{} compute the matrix of dot\PYZhy{}products with guide features}
             \PY{n}{dst}\PY{o}{.}\PY{n}{diff}\PY{p}{[}\PY{l+m+mi}{0}\PY{p}{]}\PY{o}{.}\PY{n}{reshape}\PY{p}{(}\PY{n}{ch}\PY{p}{,}\PY{o}{\PYZhy{}}\PY{l+m+mi}{1}\PY{p}{)}\PY{p}{[}\PY{p}{:}\PY{p}{]} \PY{o}{=} \PY{n}{y}\PY{p}{[}\PY{p}{:}\PY{p}{,}\PY{n}{A}\PY{o}{.}\PY{n}{argmax}\PY{p}{(}\PY{l+m+mi}{1}\PY{p}{)}\PY{p}{]} \PY{c+c1}{\PYZsh{} select ones that match best}
         
         \PY{n}{\PYZus{}}\PY{o}{=}\PY{n}{deepdream}\PY{p}{(}\PY{n}{net}\PY{p}{,} \PY{n}{img}\PY{p}{,} \PY{n}{end}\PY{o}{=}\PY{n}{end}\PY{p}{,} \PY{n}{objective}\PY{o}{=}\PY{n}{objective\PYZus{}guide}\PY{p}{)}
\end{Verbatim}


    \begin{center}
    \adjustimage{max size={0.9\linewidth}{0.9\paperheight}}{output_27_0.jpg}
    \end{center}
    { \hspace*{\fill} \\}
    
    \begin{Verbatim}[commandchars=\\\{\}]
3 9 inception\_3b/output (575, 1024, 3)

    \end{Verbatim}

    This way we can affect the style of generated images without using a
different training set.


    % Add a bibliography block to the postdoc
    
    
    
    \end{document}
